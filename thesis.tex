\documentclass{tut-thesis}
% package definitions
\usepackage[backend=biber,bibstyle=authoryear,citestyle=authoryear-comp]{biblatex} % bibliographies
\usepackage{placeins} % defines \FloatBarrier to keep floats in proper sections
\usepackage{paralist} % defines in-line environments for lists
\usepackage{graphicx} % enhanced support for graphics
\usepackage{microtype} % avoid badboxes with typewriter font
\usepackage{csquotes} % handles quotations
\usepackage{tikz} % 'draw' good looking graphics from within latex
\usepackage{ctable} % advanced tabular environment
\usepackage{amsmath} % math typesetting
\usepackage{amsfonts} % fonts needed in mathmode
\usepackage[chapter]{algorithm} % algorithmic environment
\usepackage{algpseudocode} % pseudocode
\usepackage[math]{blindtext} % generates pointles text for testing purposes

% preparatory content
\makeglossaries
\loadglsentries{glossary} % load glossary entries from the file glossary.tex
\addbibresource{bibliography.bib} % add bibliography resource
% some mathoperators
\DeclareMathOperator*{\argmin}{arg\,min}
\DeclareMathOperator*{\argmax}{arg\,max}

% macros with two arguments have the form
% \macro[secondary language]{main language}
% where the secondary language is optional
\author 	{Author Name}
\title 		[Suomenkielinen otsikko]   {English Title}
\thesistype [Diplomityö] 			   {Master of Science Thesis}
\keywords	[avainsanat] 			   {keywords}
\programme  [Koulutusohjelma] 		   {Degree Programme}
\major		[Pääaine] 				   {Major}
\examiner 	[professori Ensimmäinen Tarkastaja \and TkT Toinen Tarkastaja]
			{Professor First Examiner \and PhD Another Examiner}
\extrainfo 	{Examiner and topic approved by the
			 Faculty Council of the Faculty of
			 XXX on d~Month~Year}
\date 		{\Today} % in the final version this should be set to the date after which no changes were made e.g. \DTMDate{2016-04-07}
\location 	{Tampere}

\begin{document}
\maketitle

\frontmatter
\begin{abstract}
	Abstract goes here.
\end{abstract}

\begin{otherlanguage}{finnish}
	\begin{abstract}
		Suomalainen tiivistelmä jos sellanen tarvitaan.
	\end{abstract}
\end{otherlanguage}

\begin{preface}
	Preface goes here.
\end{preface}

\tableofcontents
\listoftables
\listoffigures
\listofalgorithms

% if your glossary has long entries, you can adjust the entrywidth
% using \setlength{\glsentrywidth}{<width>}, where <width> is e.g. 7em
\printglossary

\mainmatter
\blinddocument % generates a full document of non-sense
\ctable[%
	caption=Comparison of exotic meet prices,
	label=tab:comparisontable,
	mincapwidth=\textwidth
	]%
	{llr}{}{%  
	\toprule
	\multicolumn{2}{c}{Item} \\
	\cmidrule(r){1-2}
	Animal    & Description & Price (\$) \\
	\midrule
	Gnat      & per gram    & 13.65      \\
	      	  &    each     & 0.01       \\
	Gnu       & stuffed     & 92.50      \\
	Emu       & stuffed     & 33.33      \\
	Armadillo & frozen      & 8.99       \\
	\bottomrule
	}
\blindtext % more non-sense

We can later reference the Table~\ref{tab:comparisontable} just so~\autocite{Doe2015}. \blindtext % more non-sense

\begin{figure}
	\centering
	\begin{tikzpicture}
		\node[draw,rectangle,minimum width=.7\textwidth,minimum height=.4\textwidth] {A dummy figure};
	\end{tikzpicture}
	\caption{A dummy figure made with TikZ.}
	\label{fig:dummyfigure}
\end{figure}

\begin{algorithm}
	\caption{Some pseudocode}
	\begin{algorithmic}[1]
		\Procedure{writingThesis}{subject} \Comment{A simple procedure really}
			\State knowledge $\gets$ \Call{doResearch}{subject}

			\While{thesis $\not=$ finished}
				\State \Call{write}{thesis}
			\EndWhile
			\State\Return{thesis}
		\EndProcedure
	\end{algorithmic}
\end{algorithm}

\blindtext % more non-sense

According to \textcite{Doe2015} the Figure~\ref{fig:dummyfigure} can be referenced using the \LaTeX{} command \verb|Figure~\ref{fig:dummyfigure}|.
\blindmathpaper % non-sense with math

% all entries that are cited automatically appear in the bibliography
\printbibliography

\begin{appendices}
\chapter{Appendix}
\blindtext % again some non-sense
\chapter{Another appendix}
\blindtext
\end{appendices}

\end{document}